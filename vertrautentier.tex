\documentclass{article}
%[showframe]
\usepackage[
        a4paper,
        left=4cm,
        right=4cm,
        top=3cm,
        bottom=0.5cm,
        %showframe %use this option in order to show the "frame" it will become more clear if you use it and helps you to understand LaTeX-Geometry
]{geometry}
\setlength{\voffset}{-0.75in}
\setlength{\headsep}{5pt}
\usepackage{amssymb}
\usepackage{array}
\usepackage{tabularx}
\usepackage[pdfstartview=FitH]{hyperref} %this is the most important package here. I advice you to read its documentation https://ctan.org/pkg/hyperref?lang=en
\usepackage{xcolor}
\usepackage[ngerman]{babel}
\usepackage{wasysym}
\hypersetup{ % this is just a small setup that sets the colors to black. The default is very ugly. Remove it to check it out :)
  pdfauthor   = {Xaver Stiensmeier},
  pdfkeywords = {Xaver Stiensmeier, Vertrautentiere},
  pdftitle    = {Vertrautentiere},
  colorlinks,
  citecolor=black,
  filecolor=black,
  linkcolor=black,
  urlcolor=black
}
%\renewcommand\LayoutCheckField[2]{#1#2}
%\fboxsep=0pt
% Some new form elements are coming...
% name of the form element is always the last parameter
\newcommand{\tickbox}[1]{\rule[6pt]{0pt}{6pt}\hbox{\CheckBox[print,name=#1, width=0.7em, height=0.7em, bordercolor=0 0 0, borderwidth=0.6]{}}}
% a simple CheckBox param1 is the name. Adding it to the hbox has a very positive effect! Check it out by removing the box (don't forget to remove the curly bracket at the end as well
\newcommand{\tb}[2]{\rule[6pt]{0pt}{6pt}\hbox{\TextField[name=#2,height=0.35cm, backgroundcolor={lightgray}, width=#1, bordercolor=0 0 0, borderwidth=0]{}}}
% param1 is the width, param2 is the name.
\newcommand{\tbf}[3]{\rule[6pt]{0pt}{6pt}\hbox{\TextField[name=#3,height=0.35cm, backgroundcolor={lightgray}, width=#1, charsize=#2, bordercolor=0 0 0, borderwidth=0]{}}}
% param1 is the width, param2 is charsize in pt, param3 is the name
\newcommand{\info}[1]{\rule[6pt]{0pt}{6pt}\hbox{\TextField[name=#1,height=0.35cm, backgroundcolor={lightgray}, width=1.7cm, bordercolor=0 0 0, borderwidth=0]{}}}
% this is not the most useful command. It's a medium sized tb. param1 is the name
\newcommand{\eigenschaft}[1]{\rule[6pt]{0pt}{6pt}\hbox{\TextField[name=#1,height=0.35cm, backgroundcolor={lightgray}, width=0.5cm, bordercolor=0 0 0, borderwidth=0]{}}}
% this is just a small tb. param1 is the name

% The following can be used to automatically calculate pain-threshold. It's not perfect. It uses javascript therefore it doesn't work with many viewers
%\newcommand{\lp}{\rule[6pt]{0pt}{6pt}\hbox{\TextField[name = number.1, height=0.35cm, backgroundcolor={lightgray}, width=0.5cm, bordercolor=0 0 0, borderwidth=0, format = {
%               var f = this.getField('number.1');
%               f.userName = 'first number'
%               }]{}}}
%\newcommand{\tlp}[2]{\rule[6pt]{0pt}{6pt}\hbox{\TextField[name = #1, readonly=true, height=0.35cm, backgroundcolor={lightgray}, width=0.5cm, bordercolor=0 0 0, borderwidth=0, format = {
%               var f = this.getField('#1');
%               f.userName = '#1'
%               },
%           calculate = {
%               this.getField('#1').value =
%                 this.getField('number.1').value*#2;
%               }]{}}}
\newcolumntype{S}{>{\centering\arraybackslash}m{1.5em}} % you probably should read the documentation, but it is definetly not vital here!
%\renewcommand{\tabularxcolumn}[1]{m{#1}} % redefine 'X' to use 'm'
\newcommand{\optional}[1]{} % you may ignore this, but you can also think about when or why it might be used ^^
\begin{document}
\title{\vspace{-1.5cm}\href{https://ulisses-regelwiki.de/GR_Vertraute.html}{Vertrautentier}\vspace{-1.5cm}} % -vspace allows you to one-time reduce certain gaps. Remove it to see the difference
\date{} % removes the date. Remove to see the difference
\maketitle
\thispagestyle{empty} % removes the page-number. Remove it to see the difference
\begin{center}
\TextField[name=fname, height=0.5cm, backgroundcolor={lightgray}, width=5cm, bordercolor=0 0 0, borderwidth=0]{}\hspace{0.3cm}\TextField[name=fap,height=0.4cm, backgroundcolor={lightgray}, width=1cm, bordercolor=0 0 0, borderwidth=0, value=0]{}
\end{center}
\begin{Form}
\noindent
\makebox[\textwidth]{%
\begin{tabularx}{1.2\textwidth}{X|S|X|S|X|S|X|S} % you might wanna read into tabularx. Not vital but helpful https://ctan.org/pkg/tabularx?lang=en
   \textbf{Größe} & \tb{0.65cm}{fsize} & \textbf{Gewicht} & \tb{0.65cm}{fweight} & \textbf{Alter} & \eigenschaft{fage} & \textbf{Geschlecht} & \eigenschaft{fsex} \\
\end{tabularx}%
}\vspace{0.5cm}
\makebox[\textwidth]{%
\begin{tabularx}{1.2\textwidth}{X|S|X|S|X|S|X|S}
  \hline
  \textbf{MU} & \eigenschaft{fcourage} & \textbf{KL} & \eigenschaft{fsagacity} & \textbf{IN} & \eigenschaft{fintuition} & \textbf{CH} & \eigenschaft{fcharisma} \\
  \textbf{FF} & \eigenschaft{fdexterity} & \textbf{GE} & \eigenschaft{fagility} & \textbf{KO} & \eigenschaft{fconstitution} & \textbf{KK} & \eigenschaft{fstrength}
\end{tabularx}%
}\vspace{0.3cm}
\makebox[\textwidth]{%
\begin{tabularx}{1.2\textwidth}{X|S|X|S|X|S|X|S}
  \textbf{LeP} & \eigenschaft{flep} & \textbf{AsP} & \eigenschaft{fasp} & \textbf{INI} & \eigenschaft{fini} & \textbf{VW} & \eigenschaft{fdef} \\
  \textbf{SK} & \eigenschaft{fspirit} & \textbf{ZK} & \eigenschaft{ftoughness} & \textbf{GS} & \eigenschaft{fmovement} & \textbf{RS} & \eigenschaft{fpro}
\end{tabularx}%
}
\makebox[\textwidth]{%fw
\begin{tabularx}{1.2\textwidth}{X|X|X|X}
  \hline
  \textbf{Name} & \textbf{Attackewert} & \textbf{Trefferpunkte} & \textbf{Reichweite} \\
  \tb{3.5cm}{fwn1} & \tb{3.5cm}{fwat1} & \tb{3.5cm}{fwtp1} & \tb{3.5cm}{fwr1} \\
  \tb{3.5cm}{fwn2} & \tb{3.5cm}{fwat2} & \tb{3.5cm}{fwtp2} & \tb{3.5cm}{fwr2}
\end{tabularx}%
}
\subsubsection*{\href{https://ulisses-regelwiki.de/vor-und-nachteile.html}{Vorteile/Nachteile}}
\makebox[\textwidth]{%fvn
\begin{tabularx}{1.2\textwidth}{X|X|X|X}
  \tbf{3.5cm}{6pt}{fvn1} & \tbf{3.5cm}{6pt}{fvn2} & \tbf{3.5cm}{6pt}{fvn3} & \tbf{3.5cm}{6pt}{fvn4}\\
  \tbf{3.5cm}{6pt}{fvn5} & \tbf{3.5cm}{6pt}{fvn6} & \tbf{3.5cm}{6pt}{fvn7} & \tbf{3.5cm}{6pt}{fvn8}\\
  \tbf{3.5cm}{6pt}{fvn9} & \tbf{3.5cm}{6pt}{fvn10} & \tbf{3.5cm}{6pt}{fvn11} & \tbf{3.5cm}{6pt}{fvn12}\\
  \tbf{3.5cm}{6pt}{fvn13} & \tbf{3.5cm}{6pt}{fvn14} & \tbf{3.5cm}{6pt}{fvn15} & \tbf{3.5cm}{6pt}{fvn16}
\end{tabularx}%
}
\subsubsection*{\href{https://ulisses-regelwiki.de/sonderfertigkeiten.html}{Sonderfertigkeiten}} % one might consider to actually add all skills here
\makebox[\textwidth]{%fs
\begin{tabularx}{1.2\textwidth}{X|X|X|X}
  \tbf{3.5cm}{6pt}{fs1} & \tbf{3.5cm}{6pt}{fs2} & \tbf{3.5cm}{6pt}{fs3} & \tbf{3.5cm}{6pt}{fs4}\\
  \tbf{3.5cm}{6pt}{fs5} & \tbf{3.5cm}{6pt}{fs6} & \tbf{3.5cm}{6pt}{fs7} & \tbf{3.5cm}{6pt}{fs8}\\
  \tbf{3.5cm}{6pt}{fs9} & \tbf{3.5cm}{6pt}{fs10} & \tbf{3.5cm}{6pt}{fs11} & \tbf{3.5cm}{6pt}{fs12}\\
  \tbf{3.5cm}{6pt}{fs13} & \tbf{3.5cm}{6pt}{fs14} & \tbf{3.5cm}{6pt}{fs15} & \tbf{3.5cm}{6pt}{fs16}
\end{tabularx}%
}
\subsubsection*{\href{https://ulisses-regelwiki.de/talentauswahl.html}{Talente}}
\makebox[\textwidth]{%ft
\begin{tabularx}{1.2\textwidth}{X|X|X|X}
  \tbf{3.5cm}{6pt}{ft1} & \tbf{3.5cm}{6pt}{ft2} & \tbf{3.5cm}{6pt}{ft3} & \tbf{3.5cm}{6pt}{ft4}\\
  \tbf{3.5cm}{6pt}{ft5} & \tbf{3.5cm}{6pt}{ft6} & \tbf{3.5cm}{6pt}{ft7} & \tbf{3.5cm}{6pt}{ft8}\\
  \tbf{3.5cm}{6pt}{ft9} & \tbf{3.5cm}{6pt}{ft10} & \tbf{3.5cm}{6pt}{ft11} & \tbf{3.5cm}{6pt}{ft12}\\
  \tbf{3.5cm}{6pt}{ft13} & \tbf{3.5cm}{6pt}{ft14} & \tbf{3.5cm}{6pt}{ft15} & \tbf{3.5cm}{6pt}{ft16}
\end{tabularx}%
}
\subsubsection*{\href{https://ulisses-regelwiki.de/VSF_Vertrautentricks.html}{Vertrautentricks}}
\makebox[\textwidth]{%fv
\begin{tabularx}{1.2\textwidth}{XS|XS|XS}
  \textbf{\href{https://ulisses-regelwiki.de/VSF_Diebstahl.html}{Diebstahl}} & \tickbox{fvdiebstahl} & \textbf{\href{https://ulisses-regelwiki.de/VSF_Dingeerspueren.html}{Dinge erspüren}} & \tickbox{fvdingeerspuren} & \textbf{\href{https://ulisses-regelwiki.de/VSF_Einschuechterndes_Bellen.html}{Einschüchterndes Bellen}} & \tickbox{fveinschuchterndesbellen} \\\hline
  \textbf{\href{https://ulisses-regelwiki.de/VSF_ErsterunterGleichen.html}{Erster unter Gleichen}} & \tickbox{fversteruntergleichen} & \textbf{\href{https://ulisses-regelwiki.de/VSF_Fluchbote.html}{Fluchbote}} & \tickbox{fvfluchbote} & \textbf{\href{https://ulisses-regelwiki.de/VSF_Furchtbote.html}{Furchtbote}} & \tickbox{fvfurchtbote} \\\hline
  \textbf{\href{https://ulisses-regelwiki.de/VSF_Gaukeltricks.html}{Gaukeltricks}} & \tickbox{fvgaukeltricks} & \textbf{\href{https://ulisses-regelwiki.de/VSF_Gestalt.html}{Gefährte in neuer Gestalt}} & \tickbox{fvgefahrteinneuergestalt} & \textbf{\href{https://ulisses-regelwiki.de/VSF_Halluzinationsgift.html}{Halluzinationsgift}} & \tickbox{fvhalluzinationsgift} \\\hline
  \textbf{\href{https://ulisses-regelwiki.de/VSF_Hexensinn.html}{Hexen/Geodensinn}} & \tickbox{fvhexensinn} & \textbf{\href{https://ulisses-regelwiki.de/VSF_Hexenstimme.html}{Hexen/Geodenstimme}} & \tickbox{fvhexenstimme} & \textbf{\href{https://ulisses-regelwiki.de/VSF_HypnotischerBlick.html}{Hypnotischer Blick}} & \tickbox{fvhypnotischerblick} \\\hline
  \textbf{\href{https://ulisses-regelwiki.de/VSF_Katzenplage.html}{Katzenplage}} & \tickbox{fvkatzenplage} & \textbf{\href{https://ulisses-regelwiki.de/VSF_Kraftraub.html}{Kraftraub}} & \tickbox{fvkraftraub} & \textbf{\href{https://ulisses-regelwiki.de/VSF_Kroetengift.html}{Krötengift}} & \tickbox{fvkrötengift} \\\hline
  \textbf{\href{https://ulisses-regelwiki.de/VSF_Kroetenschlag.html}{Krötenschlag}} & \tickbox{fvkrötenschlag} & \textbf{\href{https://ulisses-regelwiki.de/VSF_Leben_aufspueren.html}{Leben aufspüren}} & \tickbox{fvlebenaufspuren} & \textbf{\href{https://ulisses-regelwiki.de/VSF_Lebensraub.html}{Nächtlicher Lebensraub}} & \tickbox{fvnachtlicherlebensraub} \\\hline
  \textbf{\href{https://ulisses-regelwiki.de/VSF_SchlafenderFreund.html}{Schlafender Freund}} & \tickbox{fvschalfenderfreund} & \textbf{\href{https://ulisses-regelwiki.de/VSF_Streicheleinheiten.html}{Streicheleinheiten}} & \tickbox{fvstreicheleinheiten} & \textbf{\href{https://ulisses-regelwiki.de/VSF_Spinnensinn.html}{Spinnensinn}} & \tickbox{fvspinnensinn} \\\hline
  \textbf{\href{https://ulisses-regelwiki.de/VSF_Stimmungssinn.html}{Stimmungssinn}} & \tickbox{fvstimmungssinn} & \textbf{\href{https://ulisses-regelwiki.de/VSF_Tiersinne.html}{Tiersinne}} & \tickbox{fvtiersinne} & \textbf{\href{https://ulisses-regelwiki.de/VSF_Teleportationsflug.html}{Teleportationsflug}} & \tickbox{fvteleportationsflug} \\\hline
  \textbf{\href{https://ulisses-regelwiki.de/VSF_Tarnung.html}{Tarnung}} & \tickbox{fvtarnung} & \textbf{\href{https://ulisses-regelwiki.de/VSF_UngesehenerBeobachter.html}{Ungesehener Beobachter}} & \tickbox{fvungesehenerbeobachter} & \textbf{\href{https://ulisses-regelwiki.de/VSF_WachsameAugen.html}{Wachsame Augen}} & \tickbox{fvwachsameaugen} \\\hline
  \textbf{\href{https://ulisses-regelwiki.de/VSF_Zwiegespraech.html}{Zwiegespräch}} & \tickbox{fvzwiegesprach}
\end{tabularx}%
}
\end{Form}
\end{document}
